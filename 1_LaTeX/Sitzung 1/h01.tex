\documentclass[a4paper, 12pt]{article}

\usepackage[margin=2cm]{geometry}
\setlength{\parindent}{0em}
\setlength{\parskip}{\smallskipamount}
\usepackage[ngerman]{babel}
\usepackage{minted}


\begin{document}
	\pagestyle{empty}
	\begin{minipage}{0.2\textwidth}
		Marvin Ködding\\\scriptsize MAT 06\\PH Heidelberg
	\end{minipage}
	\begin{minipage}{.6\textwidth}
		\centering
		-- Vorbereitungsblatt 01 --\\\textbf{Grundlagen zu \LaTeX}
	\end{minipage}
	\begin{minipage}{.2\textwidth}
		\flushright SoSe 2025\\\scriptsize 22.04.2025
	\end{minipage}
	
	\vspace*{1cm}
	
	\textbf{Aufgabe 1}
	
	Legt eine \LaTeX-Datei an. Erzeugt ein Dokument und bearbeitet die folgenden Aufgaben.
	
	\vspace*{0.4cm}
	
	\textbf{Aufgabe 2}
	
		Setzt den folgenden Text:
		
		\begin{center}
			Mathematik digital kompetent\\
			\tiny Marvin Ködding \qquad PH Heidelberg
		\end{center}
		
		\textbf{Zusammenfassung.} Wir werden uns im folgenden Semester mit den Themen
		\begin{itemize}
			\item \LaTeX,
			\item \textsc{Python} und
			\item \glqq{}Manim\grqq{}
		\end{itemize}
		beschäftigen. Dabei legen wir einen besonderen Fokus auf die \textbf{mathematische} Exploration mit \textsc{Computern}.
		
	\vspace*{0.4cm}
	
	\textbf{Aufgabe 3}
	
		Setzt die folgende Tabelle:
		
		\begin{center}
			\begin{tabular}{c|l}
				Text 1 & Nummer 1\\
				\hline
				2 & 2\\
				\hline
				3 & 3
			\end{tabular}
		\end{center}
		
	\vspace*{0.4cm}

	\textbf{Aufgabe 4}
	
		Setzt den folgenden Text mit Python-Syntaxhighlighting:
		
		\begin{minted}{python}
	b = 10
	if b > 20:
	  print("b > 20")
	print("b =", b)
		\end{minted}
\end{document}