\documentclass[a4paper, 12pt]{mksheetphhd}

\title{Grundlagen von \LaTeX}
\author{Marvin Ködding}
%\usepackage{minted}


\begin{document}
	\printtitle
	
	\vspace*{.5cm}
	
	\begin{aufgabe}
		Legt eine \LaTeX-Datei an. Erzeugt ein Dokument und setzt das Vorbereitungsblatt 01.
		
		Hinweise:
		\begin{itemize}
			\item mit \verb|\setlength{\parindent}{0em}| könnt ihr die Einrückung eines neuen Absatzes auf \verb*|0em| setzen. \verb*|em| ist in \LaTeX{} eine Längeneinheit und beschreibt die Breite des Buchstabens \verb*|m|.
			\item mit \verb|\setlength{\parskip}{\medskipamount}| kann man den Abstand zwischen zwei Absätzen auf die Länge von \verb*|\medskip| setzen.
			\item Man kann verschiedene Arten von waagerechten Strichen setzen:
			\begin{itemize}
				\item[] \verb|-| wird zu -,
				\item[] \verb|--| wird zu -- und
				\item[] \verb|---| wird zu ---.
			\end{itemize}
			\item Mit \verb|\pagestyle{empty}| kann man die Seitenzahl entfernen.
		\end{itemize}
		
	\end{aufgabe}
\end{document}