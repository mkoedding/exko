\documentclass[a4paper, 12pt]{mksheetphhd}

\title{Grundlagen von \LaTeX\\\small Handwerkzeug}
\author{Marvin Ködding}
%\usepackage{minted}
\usepackage{multicol}

\everymath{\displaystyle}

\begin{document}
	\printtitle
	
	\vspace*{.5cm}
	
	\begin{aufgabe}
		Setzt die folgenden Ausdrücke:
		\begin{enumerate}
			\begin{multicols}3
				\item $\alpha \leq \beta$
				\item $A \subseteq B$
				\item $x_1^2 + x_2^2 = r^2$
				\item $x + \sqrt{2}$
				
				
				\item $\mathbb{N} \subset \mathbb{Z} \subset \mathbb{Q} \subset \mathbb{R} \subset \mathbb{C}$
				
				\item $\lim_{n \to \infty}\frac{1}{n^2}$
				
				\item $\sin^2x+\cos^2x = 1$
				
				\item $\sum_{i = 1}^{n} i^2 = \frac{n(n+1)(n+2)}{6}$
				
				\item 
				
			\end{multicols}
		\end{enumerate}
	\end{aufgabe}
	
	\begin{aufgabe}
		Bestimmt die Lösungen der Gleichung $3x^2+6x=12$ schriftlich. Stelle deine Umformungen mit \LaTeX{} dar. Setze die Gleichheitszeichen untereinander. (Hinweis: \verb+align*+)
	\end{aufgabe}
	
	\begin{aufgabe}
		Setzt den folgenden Text:
		
		Die Menge der ganzen Zahlen ist $\mathbb{Z} = \mathbb{N} \cup \{0\} \cup \{-n \mid n \in \mathbb{N}\}$. 
		
		(Hinweis: Mengenklammern setzt man mit \verb|\{ \}|, die Mengen $\mathbb{Z}$ und $\mathbb{N}$ setzt man mit \verb|\mathbb{Z}| bzw. \verb|\mathbb{N}|.)
	\end{aufgabe}
	
	
\end{document}