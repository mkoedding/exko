\documentclass[a4paper, 12pt]{mksheetphhd}

\title{Grundlagen von \LaTeX\\\small Matrizen und Darstellung}
\author{Marvin Ködding}
%\usepackage{minted}
\usepackage{multicol}

\everymath{\displaystyle}

\begin{document}
	\printtitle
	
	\vspace*{.5cm}
	
	\begin{aufgabe}
		Setzt die folgenden Ausdrücke:
		\begin{enumerate}
			\begin{multicols}3
				\item $\alpha \leq \beta$
				\item $A \subseteq B$
				\item $x_1 + x_2 = z$
				\item $x + \sqrt{2}$
				\item $A = \begin{pmatrix}
					1 & 0 \\ 0 & 1
				\end{pmatrix}$
				\item $\mathcal{Z} = \begin{Vmatrix}
					\alpha & 0 & 0\\ 1 & \beta & 0 \\ 0 & 1 & \gamma
				\end{Vmatrix}$
				\item $\mathcal{Z} = \begin{bmatrix}
					1 & 0 & 1\\ 0 & 1 & 0 \\ 1 & 0 & 1
				\end{bmatrix}$
			\end{multicols}
		\end{enumerate}
	\end{aufgabe}
	
	\begin{aufgabe}
		Macht euch den Unterschied zwischen \verb+\int_a^b+ und \verb+\int\limits_a^b+ klar.
	\end{aufgabe}
	
	\begin{aufgabe}
		Setzt den folgenden Text:
		
		Den \textbf{Satz des Pythagoras} können wir in einem \textit{rechtwinkligen} Dreieck mit den Kathetenlängen $a$ und $b$ und der Hypothenusenlänge $c$ anwenden. Dann gilt \[a^2+b^2=c^2.\]
	\end{aufgabe}
	
	
\end{document}