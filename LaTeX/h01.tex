\documentclass[a4paper, 12pt]{article}


\usepackage{amsmath, amsthm, amsfonts}
\usepackage[margin=2cm]{geometry}
\setlength{\parindent}{0em}
\setlength{\parskip}{\medskipamount}
%\usepackage{bera}
%\usepackage{euler}
\usepackage{hyperref}
\usepackage[ngerman]{babel}
\usepackage{xcolor}
\definecolor{pinegreen}{cmyk}{0.92,0,0.59,0.25} % PineGreen
\definecolor{craneorange}{RGB}{252,187,6}
\usepackage{tikz, tikz-timing}
\usetikztiminglibrary{clockarrows}
\usetikzlibrary[arrows,shadows.blur,matrix,calc,decorations.footprints]
\tikzstyle{to}=[->, >=stealth]
\usepackage{comment}
\usepackage{minted}


\renewcommand{\date}{$\leq$22.04.2025}
\newcommand{\klasse}{MAT 06}
\newcommand{\institute}{PH Heidelberg}
\newcommand{\topic}{Grundlagen zu \LaTeX}
\newcommand{\semester}{SoSe 2025}

\newcommand{\blatt}{01}

\newcounter{Aufgabe}
\newenvironment{aufgabe}{\addtocounter{Aufgabe}{1}%%
	\textbf{Aufgabe \theAufgabe}
	
	\vspace*{-0.3em}}{\medskip}

\newcommand{\loes}[1]{
	%\textcolor{blue}{#1}
}

\begin{document}
	\pagestyle{empty}
	\begin{minipage}{0.2\textwidth}
		Marvin Ködding\\\scriptsize \klasse\\\institute
	\end{minipage}
	\begin{minipage}{.6\textwidth}
		\centering
		-- Vorbereitungsblatt \blatt{} --\\\textbf{\topic}
	\end{minipage}
	\begin{minipage}{.2\textwidth}
		\flushright \semester\\\scriptsize \date
	\end{minipage}
	
	\vspace*{1cm}
	
	\begin{aufgabe}
		Legt eine \LaTeX-Datei an. Erzeugt ein Dokument und bearbeitet die folgenden Aufgaben.
	\end{aufgabe}
	
	\begin{aufgabe}
		Setzt den folgenden Text:
		
		\begin{center}
			Kontextuale Grammatiken\\
			\tiny Marvin Ködding \qquad PH Heidelberg
		\end{center}
		
		\textbf{Zusammenfassung.} \textsf{Kontextuale Grammatiken} wurden von \textsc{Solomon Marcus} in seiner Arbeit \textit{Contextual Grammars} als formales Modell für die Modellierung von \glqq{}natürlicher Sprache\grqq{} eingeführt.
	\end{aufgabe}
	
	\begin{aufgabe}
		Setzt die folgende Tabelle:
		
		\begin{center}
			\begin{tabular}{c|l}
				Text 1 & Nummer 1\\
				\hline
				2 & 2\\
				\hline
				3 & 3
			\end{tabular}
		\end{center}
	\end{aufgabe}
	
	\begin{aufgabe}
		Setzt den folgenden Text mit Python-Syntaxhighlighting:
		
		\begin{minted}{python}
b = 10
  if b > 20:
  print("b > 20")
print("b =", b)
		\end{minted}
	\end{aufgabe}
\end{document}