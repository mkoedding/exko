\documentclass[a4paper, 12pt]{article}


\usepackage{amsmath, amsthm, amsfonts}
\usepackage[margin=2cm]{geometry}
\usepackage{multicol}
\setlength{\parindent}{0em}
\setlength{\parskip}{\medskipamount}
%\usepackage{bera}
%\usepackage{euler}
\usepackage{hyperref}
\usepackage[ngerman]{babel}
\usepackage{xcolor}
\definecolor{pinegreen}{cmyk}{0.92,0,0.59,0.25} % PineGreen
\definecolor{craneorange}{RGB}{252,187,6}
\usepackage{tikz, tikz-timing}
\usetikztiminglibrary{clockarrows}
\usetikzlibrary[arrows,shadows.blur,matrix,calc,decorations.footprints]
\tikzstyle{to}=[->, >=stealth]
\usepackage{comment}
%\usepackage{minted}


\renewcommand{\date}{$\leq$29.04.2025}
\newcommand{\klasse}{MAT 06}
\newcommand{\institute}{PH Heidelberg}
\newcommand{\topic}{Grundlagen zu \LaTeX}
\newcommand{\semester}{SoSe 2025}

\newcommand{\blatt}{02}

\newcounter{Aufgabe}
\newenvironment{aufgabe}{\addtocounter{Aufgabe}{1}%%
	\textbf{Aufgabe \theAufgabe}
	
	\vspace*{-0.3em}}{\medskip}

\newcommand{\loes}[1]{
	%\textcolor{blue}{#1}
}

\everymath{\displaystyle}

\begin{document}
	\pagestyle{empty}
	\begin{minipage}{0.2\textwidth}
		Marvin Ködding\\\scriptsize \klasse\\\institute
	\end{minipage}
	\begin{minipage}{.6\textwidth}
		\centering
		-- Vorbereitungsblatt \blatt{} --\\\textbf{\topic}
	\end{minipage}
	\begin{minipage}{.2\textwidth}
		\flushright \semester\\\scriptsize \date
	\end{minipage}
	
	\vspace*{1cm}
	
	\begin{aufgabe}
		Setzt die folgenden Ausdrücke:
		\begin{enumerate}
			\begin{multicols}3
				\item $\alpha \leq \beta$
				\item $A \subseteq B$
				\item $x_1 + x_2 = z$
				\item $x + \sqrt{2}$
				\item $\frac{a\cdot b}{2}$
				\item ${n \choose k}$
				\item $A = \begin{pmatrix}
					1 & 0 \\ 0 & 1
				\end{pmatrix}$
				\item $\mathbb{N} \subset \mathbb{Z} \subset \mathbb{Q} \subset \mathbb{R} \subset \mathbb{C}$
				\item $
				\underbrace{1 + 1}_{=\;2} \leq \underbrace{1 + 2 + 3}_{=\;6}
				$
			\end{multicols}
		\end{enumerate}
	\end{aufgabe}
	
	\begin{aufgabe}
		Verwendet die folgenden \LaTeX{}-Ausdrücke und vergleicht diese.
		\begin{itemize}
			\item \verb|\[(\int_a^b x^2 \mathrm{d}x)\]|
			\item \verb|\[\left(\int_a^b x^2 \mathrm{d}x \right)\]|
		\end{itemize}
	\end{aufgabe}
	
	\begin{aufgabe}
		Macht euch den Unterschied zwischen \verb+\int_a^b+ und \verb+\int\limits_a^b+ klar.
	\end{aufgabe}
	
	\begin{aufgabe}
		Setze die folgenden Formeln:
		$$C_{\alpha}=\int\limits_0^{\alpha} \Gamma(x) \mathrm{d}x$$
	\end{aufgabe}
	
	
\end{document}