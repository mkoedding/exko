\documentclass[a4paper, 12pt]{article}


\usepackage{amsmath, amsthm, amsfonts}
\usepackage[margin=2cm]{geometry}
\setlength{\parindent}{0em}
\setlength{\parskip}{\medskipamount}
%\usepackage{bera}
%\usepackage{euler}
\usepackage{hyperref}
\usepackage[ngerman]{babel}
\usepackage{xcolor}
\definecolor{pinegreen}{cmyk}{0.92,0,0.59,0.25} % PineGreen
\definecolor{craneorange}{RGB}{252,187,6}
\usepackage{tikz, tikz-timing}
\usetikztiminglibrary{clockarrows}
\usetikzlibrary[arrows,shadows.blur,matrix,calc,decorations.footprints]
\tikzstyle{to}=[->, >=stealth]
\usepackage{comment}
\usepackage{minted}


\renewcommand{\date}{22.04.2025}
\newcommand{\klasse}{MAT 06}
\newcommand{\institute}{PH Heidelberg}
\newcommand{\topic}{Grundlagen zu \LaTeX}
\newcommand{\semester}{SoSe 2025}

\newcommand{\blatt}{01}

\newcounter{Aufgabe}
\newenvironment{aufgabe}{\addtocounter{Aufgabe}{1}%%
	\textbf{Aufgabe \theAufgabe}
	
	\vspace*{-0.3em}}{\medskip}

\newcommand{\loes}[1]{
	%\textcolor{blue}{#1}
}

\begin{document}
	\pagestyle{empty}
	\begin{minipage}{0.2\textwidth}
		Marvin Ködding\\\scriptsize \klasse\\\institute
	\end{minipage}
	\begin{minipage}{.6\textwidth}
		\centering
		-- Präsenzblatt \blatt{} --\\\textbf{\topic}
	\end{minipage}
	\begin{minipage}{.2\textwidth}
		\flushright \semester\\\scriptsize \date
	\end{minipage}
	
	\vspace*{1cm}
	
	\begin{aufgabe}
		Legt eine \LaTeX-Datei an. Erzeugt ein Dokument mit Titel, Autor, mehreren Abschnitten, etwas Text und einer Fußnote.
	\end{aufgabe}
	
	\begin{aufgabe}
		Setzt die folgende Tabelle:
		
		\begin{center}
			\begin{tabular}{|l||c|r|}
				\hline
				\textsc{Nummer} & \textsc{Text} & \textit{Buchstabe}\\
				\hline
				\textbf{1} & links & x\\
				\textbf{2} & gerade & y\\
				\textbf{3} & mitte & z\\
				\hline
			\end{tabular}
		\end{center}
	\end{aufgabe}
	
	\begin{aufgabe}
		Erstellt eine (un-)nummerierte Liste mit 3 Punkten. Baut nun unter dem 2. Punkt eine weitere (un-)nummerierte Liste ein und untersucht das Verhältnis von \glqq{}Oberliste\grqq{} und \glqq{}Unterliste\grqq{}.
	\end{aufgabe}
	
	\begin{aufgabe}
		Nun wollen wir Zitate in euren Text einbinden. Dafür erstellen wir zunächst eine \verb*|literature.bib|-Datei. Danach erzeugen wir einen Literatureintrag\footnote{Entweder schreibt ihr diesen händisch oder (was ich empfehlen würde) ihr verwendet Zotero oder Citavi und exportiert eure Literaturliste.}. Nun könnt ihr auf diesen Literatureintrag mit \verb*|\cite| verweisen.
	\end{aufgabe}
\end{document}