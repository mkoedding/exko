\documentclass[a4paper, 12pt]{article}


\usepackage{amsmath, amsthm, amsfonts}
\usepackage[margin=2cm]{geometry}
\usepackage{multicol}
\setlength{\parindent}{0em}
\setlength{\parskip}{\medskipamount}
%\usepackage{bera}
%\usepackage{euler}
\usepackage{hyperref}
\usepackage[ngerman]{babel}
\usepackage{xcolor}
\definecolor{pinegreen}{cmyk}{0.92,0,0.59,0.25} % PineGreen
\definecolor{craneorange}{RGB}{252,187,6}
\usepackage{tikz, tikz-timing}
\usetikztiminglibrary{clockarrows}
\usetikzlibrary[arrows,shadows.blur,matrix,calc,decorations.footprints]
\tikzstyle{to}=[->, >=stealth]
\usepackage{comment}
%\usepackage{minted}


\renewcommand{\date}{29.04.2025}
\newcommand{\klasse}{MAT 06}
\newcommand{\institute}{PH Heidelberg}
\newcommand{\topic}{Grundlagen zu \LaTeX}
\newcommand{\semester}{SoSe 2025}

\newcommand{\blatt}{02}

\newcounter{Aufgabe}
\newenvironment{aufgabe}{\addtocounter{Aufgabe}{1}%%
	\textbf{Aufgabe \theAufgabe}
	
	\vspace*{-0.3em}}{\medskip}

\newcommand{\loes}[1]{
	%\textcolor{blue}{#1}
}

\everymath{\displaystyle}

\begin{document}
	\pagestyle{empty}
	\begin{minipage}{0.2\textwidth}
		Marvin Ködding\\\scriptsize \klasse\\\institute
	\end{minipage}
	\begin{minipage}{.6\textwidth}
		\centering
		-- Präsenzblatt \blatt{} --\\\textbf{\topic}
	\end{minipage}
	\begin{minipage}{.2\textwidth}
		\flushright \semester\\\scriptsize \date
	\end{minipage}
	
	\vspace*{1cm}
	
	\begin{aufgabe}
		Setze den folgenden Text:
		
		Den \textbf{Satz des Pythagoras} können wir in einem \textit{rechtwinkligen} Dreieck mit den Kathetenlängen $a$ und $b$ und der Hypothenusenlänge $c$ anwenden. Dann gilt \[a^2+b^2=c^2.\]
	\end{aufgabe}
	\begin{aufgabe}
		Setze den folgenden Text: 
		
		Mithilfe der \textbf{Mitternachtsformel} können wir die Lösungen der Gleichung $ax^2+bx+c=0$ bestimmen. Diese sind
		\[x_{1,2}=\frac{-b\pm\sqrt{b^2-4ac}}{2a}.\]
		
		
		Formuliere einen ähnlichen Text für die $pq$-Formel.
	\end{aufgabe}
	
	
	
	\begin{aufgabe}
		Bestimme die Lösungen der Gleichung $3x^2+6x=12$ schriftlich. Stelle deine Umformungen mit \LaTeX{} dar. Setze die Gleichheitszeichen untereinander. (Hinweis: \verb+align*+)
	\end{aufgabe}
	
	\begin{aufgabe}
		Setze:
		\begin{enumerate}
			\begin{multicols}2
				\item $x\lor (y \land z) = (x \land y) \lor (x \land z)$
				\item $\left(x\cdot \left(b+c\right)\right)\left(\left(a\cdot b\right)+\frac{c}{x}\right)\;\text{für } x\neq 0$
			\end{multicols}
		\end{enumerate}
	\end{aufgabe}
	
	\begin{aufgabe}
		Setzt die folgenden Ausdrücke:
		\begin{enumerate}
			\begin{multicols}2
				\item $x_1^2 + x_2^2 = r^2$
				\item $\sum_{i = 1}^{n} i^2 = \frac{n(n+1)(n+2)}{6}$
				\item $\int\limits_0^\pi \sin(x)\; \mathrm{d}x$
				\item $\lim_{x \to \infty} \frac{\ln x}{x} = 0$
			\end{multicols}
		\end{enumerate}
	\end{aufgabe}
\end{document}